\documentclass{../mfa}
\sheet{3}
\usepackage{enumitem}
\usepackage{tikz}
\usetikzlibrary{positioning,calc,intersections}
\begin{document}
\maketitle
\renewcommand{\v}[1]{\mbf{v}_{#1}}

\section{}

\subsection{}
\begin{align*}
   \varphi_\mathcal{B}(\v{i}) & = \varphi_\mathcal{B}(0\v{1} + \ldots + 1\v{i} + \ldots + 0\v{n}) \\
                        & = (0,\ldots, 1, \ldots, 0)
\end{align*}

\subsection{}

Sei $\alpha \in \mathbb{K}^n$ und $\mbf{v} = \sum_{i=1}^n \alpha_i \v{i}$ sowie $\mbf{u} = \sum_{i=1}^n \beta_i \mbf{u}$.

\newcommand{\myphi}[1]{\varphi_\mathcal{B}\left(#1\right)}
{
   \renewcommand{\v}{\mathbf{v}}
   \renewcommand{\u}{\mathbf{u}}
   \begin{align*}
      \myphi{\alpha\v + \u} & = \myphi{\alpha\sum_{i=1}^n \alpha_i \v{i} + \sum_{i=1}^n \beta_i \v{i}} \\
                            & = \myphi{\sum_{i=1}^n (\alpha\alpha_i + \beta_i)\v{i}} \\
                            & = (\alpha\alpha_1 + \beta_1 + \ldots + \alpha\alpha_n + \beta_n) \\
                            & = (\alpha\alpha_, \ldots, \alpha\alpha_n) + (\beta_, \ldots, \beta_n) \\
                            & = \alpha(\alpha_, \ldots, \alpha_n) + (\beta_, \ldots, \beta_n) \\
                            & = \alpha\myphi{\v} + \myphi{\u}
   \end{align*}
}

\subsection{}

Seien $\mbf{x} \neq \mbf{y} \in V$.
Da $\{\v{1}, \ldots, \v{n}\}$ eine Basis ist, existieren eindeutige $\alpha_i$ mit $\sum_{i=1}^n \alpha_i \v{i} =
\mbf{x}$ und eindeutige $\beta_i$. mit $\sum_{i=1}^n \alpha_i \v{i} =
\mbf{y}$.
Wenn $\mbf{x} \neq \mbf{y}$, so können nicht alle $\alpha_i = \beta_i$ sein, sonst wäre die Darstellung nicht eindeutig.
Also is die Funtkion injektiv. 

Für $(\alpha_1,\ldots,\alpha_n)$ gibt es aber auch genau ein Urbild, nämlich $\alpha_1\v{1} + \ldots + \alpha_n\v{n}$.
Sie ist also auch surjektiv, also bijektiv.

\section{}

\subsection{}

\renewcommand{\b}[1]{\mbf{b_{#1}}}
Es sei $\b{1} = \inlinepmatrix{1 \\ 1}$ und $\b{2} = \inlinepmatrix{1 \\ -1}$.
\begin{align*}
   \myphi{\v{}=(v_1,v_2)} & = \myphi{v_1 \cdot 0.5 \cdot (\b{1} + \b{2}) + v_2 \cdot 0.5 \cdot (\b{1} - \b{2})} \\
                          & = \myphi{v_1 \cdot 0.5 \cdot \b{1} + v_1 \cdot 0.5 \cdot \b{2} + v_2 \cdot 0.5 \cdot \b{1} -
v_2 \cdot 0.5 \cdot \b{2}}  \\
                          & = \myphi{(v_1 + v_2) \cdot 0.5  \cdot \b{1} + (v_1 - v_2) \cdot 0.5 \cdot \b{2}}  \\
                          & = \myphi{(v_1 + v_2) \cdot 0.5, (v_1 - v_2) \cdot 0.5}  \\
\end{align*}

\subsection{}

\begin{align*}
   \myphi{\inlinepmatrix{1 \\ 2}} &= (1.5, -0.5) \\
   \myphi{\inlinepmatrix{1 \\ 0}} &= (0.5, 0.5)
\end{align*}

\subsection{}
\begin{center}
   \begin{tikzpicture}
      \draw[thick,->,name path=stdx] (0,0) -- ++(0:3) node[below right] {Kanonische Basis};
      \draw[thick,->,name path=stdy] (0,0) -- ++(90:3);

      \draw[thick,dashed,->,name path=bx] (0,0) -- (-45:3);
      \draw[thick,dashed,->,name path=by] (0,0) -- (45:3) node[below right] {$\mathcal{B}$};

      \draw[fill=red] (1,2) coordinate (v1) circle (2pt) node[above right=1em and .5em] {$\varphi_\mathcal{B}(\v{1})=(1.5,-0.5)$};
      \draw[fill=red] (1,0) coordinate (v2) circle (2pt) node[above right=1em and .5em] {$\varphi_\mathcal{B}(\v{2})=(0.5,0.5)$};

      \draw[dotted] (v1) -- ($(45:3)!(v1)!(0,0)$);
      \draw[dotted] (v1) -- ($(-45:3)!(v1)!(0,0)$);
      \draw[dotted] (v2) -- ($(45:3)!(v2)!(0,0)$);
      \draw[dotted] (v2) -- ($(-45:3)!(v2)!(0,0)$);

   \end{tikzpicture}
\end{center}
\subsection{}

Aus 
\begin{align*}
   \frac{v_1}{2} + \frac{v_2}{2} & = a_1 \\
   \frac{v_1}{2} - \frac{v_2}{2} & = a_2 \\
\end{align*}
folgt durch elementare Umformungen $v_1 = a_1 + a_2$ und $v_2 = a_1 - a_2$, sodass
\begin{equation*}
   \varphi_\mathcal{B}^{-1}\left(\inlinepmatrix{a_1 \\ a_2}\right) = \inlinepmatrix{a_1 + a_2 \\ a_1 - a_2}.
\end{equation*}

\section{}
\subsection{}
Die Bilder der Basisvektoren $\{\v{1}, \v{2}, \v{3}\}$ sind
\begin{align*}
   D(p(x)=1) &= (p^\prime(x) = 0) = 0\v{1} + 0\v{2} + 0\v{3}\\
   D(p(x)=2) &= (p^\prime(x) = 1) = 1\v{1} + 0\v{2} + 0\v{3}\\
   D(p(x)=3) &= (p^\prime(x) = 2x) = 0\v{1} + 2\v{2} + 0\v{3}\\
\end{align*}
Die darstellende Matrix ist also
\begin{equation*}
   A = \inlinepmatrix{0 & 1 & 0 \\ 0 & 0 & 2 \\ 0 & 0 & 0}
\end{equation*}

\subsection{}
Der Kern von $D$ sind alle Polynome, deren Ableitung die Nullfunktion ist, also
alle Konstanten Polynome. Der Kern hat Dimension $1$, denn $\{p(x)=1\}$ ist eine
Basis.

\section{}

Seien $(\v{1}, \v{2}, \v{3}) = (1,x,x^2) = \mathcal{C}$
\begin{align*}
   \varphi_\mathcal{C}(a + bx +cx^2) & = \varphi_\mathcal{C}(a(\v{3} - \v{2} - (\v{1} - \v{3})) + b(\v{1} - \v{3}) + c(\v{1} + \v{2} - \v{3})) \\
                                     & = \varphi_\mathcal{C}(a(-\v{2} - \v{1} +
   2\v{3}) + b(\v{1} - \v{3}) + c(\v{1} + \v{2} - \v{3})) \\
                                     & = \varphi_\mathcal{C}(-a\v{1} + b\v{1} +
   c\v{1} -a\v{1} + c\v{2} + 2a\v{3} - b\v{3} - c\v{3}) \\
                                     & = \left(-a +b +c, -a + c, 2a -b - c\right)
\end{align*}

\subsection{}
Indem man die drei Polynome aus $\mathcal{B}$ ableitet und mithilfe der zweiten
Zeile in der obigen Definition von $\varphi_\mathcal{C}$ nach $\mathcal{C}$
übersetzt, ergibt sich:
\begin{IEEEeqnarray*}{rClL}
   D(\v{1}) & = & D(p=1) = 0    & \Rightarrow a = b = c = 0 \\
   D(\v{2}) & = & D(p=x) = 1    & \Rightarrow a = 1, b = c = 0 \\
   D(\v{3}) & = & D(p=x^2) = 2x & \Rightarrow a = c = 0, b = 2 \\
\end{IEEEeqnarray*}
\begin{equation*}
   A = \inlinepmatrix{0 & -1 & 2 \\ 0 & -1 & 0 \\ 0 & 2 & -2}
\end{equation*}
\end{document}
