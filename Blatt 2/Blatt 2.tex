\documentclass{../mfa}
\sheet{2}
\usepackage{enumitem}
\begin{document}
\maketitle
\renewcommand{\v}[1]{\mbf{v}_{#1}}
\section{}
\subsection{}
\begin{itemize}
   \item[$f$:] \begin{align*}
         f\left(A+D\right) &= f\left(\begin{pmatrix} a_1 + a_2 & b_1 + b_2 \\ 0 & c_1 + c_2
   \end{pmatrix}\right) \\
   &= \begin{pmatrix}
   (a_1 + a_2) - (c_1 + c_2) \\
   3(b_1 + b_2) \\
   -2(a_1 + a_2) + 2(c_1 + c_2)
\end{pmatrix} \\
&= \begin{pmatrix}
(a_1 - c_1) + (a_2 - c_2) \\
3b_1 + 3b_2 \\
-2a_1 + 2c_1 - 2a_2 + 2c_2
\end{pmatrix} = f\left(A) + f(D\right)
\end{align*}
$f$ ist also linear.
\item[$g$:] Es seien $p_1(x) = a_1x^2 + b_1x + c_1$ und $p_2(x) = a_2x^2 + b_2x
   + c_2$.
   \begin{align*}
      g(p_1 + p_2) &= \begin{pmatrix}
      0 & a_1 + a_2 - 2 \\
      0 & -b_1 -b_2 + c_1 + c_2
   \end{pmatrix} \\
   g(p_1) + g(p_2) &= \begin{pmatrix}
   0 & a_1 - 2 \\
   0 & -b_2 + c_1
\end{pmatrix} + \begin{pmatrix}
   0 & a_2 - 2 \\
   0 & -b_2 + c_2
\end{pmatrix} = \begin{pmatrix}
   0 & a_1 + a_2 - 4 \\
   0 & -b_1 -b_2 + c_1 + c_2
\end{pmatrix} \neq g(p_1 + p_2)
   \end{align*}
   $g$ ist also nicht linear.
\end{itemize}

\subsection{}
Um $Ker(f)$ zu berechnen, muss folgendes Gleichungssystem gelöst werden.
\begin{align*}
   a - c    & = 0 \\
   3b       & = 0 \\
   -2a + 2c & = 0 \\
   \Rightarrow b &= 0 \\
   a &= c
\end{align*}
Also $Ker(f) = \left\{k \cdot \begin{pmatrix} 1 & 0 \\ 0 & 1 \end{pmatrix} \mid k \in \mathbb{R} \right\}$

\subsection{}

Für $\begin{pmatrix} x_1 \\ x_2 \\ x_3 \end{pmatrix}\in \mathbb{R}^3$ gilt 
\begin{align*}
   a - c                                  & = x_1 \\
   3b                                     & = x_2 \\
   -2a + 2c                               & = x_3, \\
   \intertext{was sich umformen lässt zu}
   a                                      & = x_1 + c \\
   3b                                     & = x_2 \\
   -2(x_1 + c) + 2c                       & = x_3 \\
   \Rightarrow -2x_1                      & = x_3
\end{align*}
Es ist daher nicht möglich, $x_1$ und $x_3$ frei zu wählen. Der Vektor
$\begin{pmatrix} 1 & 0 & 1 \end{pmatrix}^T$ wird z.\,B. nicht getroffen. Somit
ist $f$ nicht surjektiv und auch nicht bijektiv.

Der Kern von $f$ ist $\neq \{\mbf{0}\}$. Nach Satz 1.11 ist $f$ also nicht injektiv.

\section{}
\subsection{}

\begin{proof}
   Angenommen, $\{\v{1}, \ldots, \v{n}\}$ sind linear abhängig. Somit existieren
   $\alpha_1, \ldots \alpha_n \in V$, sodass
   \begin{align*}
      \sum_{i=1}^n \alpha_i \v{i} &= \mbf{0},\\
      \intertext{mit $\exists i : \alpha_i \neq 0$. Anwendung von $f$ ergibt}
      \sum_{i=1}^n \alpha_i f(\v{i}) &= f(\mbf{0}) = \mbf{0}\\
   \end{align*}
   Da nicht $\alpha_i = 0 ~\forall i$, sind $\{f(\v{1}), \ldots, f(\v{n})\}$
   linear abhängig.
\end{proof}

\subsection{}

\begin{proof}
   Angenommen, $\{f(\v{1}), \ldots, f(\v{n})\}$ sind linear abhängig. Somit
   existieren $\beta_1, \ldots, \beta_n \in W$  mit $\exists i : \alpha_i \neq
   0$. Mithin
   \begin{align*}
      \beta_1 f(\v{1}) + \ldots + \beta_n f(\v{n}) &= 0 & | -\beta_n f(\v{n})\\
      \beta_1 f(\v{1}) + \ldots + \beta_{n-1} f(\v{n-1}) &= -\beta_n f(\v{n})\\
      f\left(\sum_{i=1}^{n-1}\beta_i \v{i}\right) &=f(-\beta_n \v{n})
   \end{align*}
   Es gilt nun, zwei Fälle zu unterscheinden.
   \begin{itemize}[align=left]
      \item[\bfseries $f$ ist injektiv:] Es gilt wegen der Injektivität
         $\sum_{i=1}^{n-1}\beta_i \v{i} = -\beta_n \v{n}$. Daher
         $\sum_{i=1}^{n}\beta_i \v{i} = \mbf{0}$, wobei nicht
         alle $\beta_i = 0$ sind. Die $\v{i}$ sind also linear abhängig.
      \item[\bfseries $\v{i}$ sind linear unabhängig:] Es gilt wie oben
         \begin{align*}
            \beta_1 f(\v{1}) + \ldots + \beta_n f(\v{n}) &= \mbf{0} \\
            f(\beta_1 \v{1} + \ldots + \beta_n \v{n}) &= \mbf{0} \\
            \intertext{mit $\beta_1 \v{1} + \ldots + \beta_n \v{n} \neq \mbf{0}$, da
            nicht alle $\beta_i = 0$ sind. Allerdings ist auch}
            f(\mbf{0}) &= \mbf{0}, \\
            \intertext{denn $f$ ist linear. $f$ kann also nicht injektiv sein,
            da zwei ungleiche $\v{i} \in V$ auf das selbe Element in $W$
         abgebildet werden.}
         \end{align*}
   \end{itemize}
   Falls also $f(\v{1}), \ldots, f(\v{n})$ linear abhängig sind, sind entweder
   $\v{1}, \ldots, \v{n}$ linear abhängig, oder $f$ ist nicht injektiv, aber
   nicht beides.
\end{proof}

\end{document}
