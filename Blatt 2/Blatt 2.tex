\documentclass{../mfa}
\sheet{2}
\usepackage{enumitem}
\begin{document}
\maketitle
\renewcommand{\v}[1]{\mbf{v}_{#1}}
\section{}
\subsection{}
\begin{itemize}
   \item[$f$:] \begin{align*}
         f\left(A+D\right) &= f\left(\begin{pmatrix} a_1 + a_2 & b_1 + b_2 \\ 0 & c_1 + c_2
   \end{pmatrix}\right) \\
   &= \begin{pmatrix}
   (a_1 + a_2) - (c_1 + c_2) \\
   3(b_1 + b_2) \\
   -2(a_1 + a_2) + 2(c_1 + c_2)
\end{pmatrix} \\
&= \begin{pmatrix}
(a_1 - c_1) + (a_2 - c_2) \\
3b_1 + 3b_2 \\
-2a_1 + 2c_1 - 2a_2 + 2c_2
\end{pmatrix} = f\left(A) + f(D\right)
\end{align*}
$f$ ist also linear.
\item[$g$:] Es seien $p_1(x) = a_1x^2 + b_1x + c_1$ und $p_2(x) = a_2x^2 + b_2x
   + c_2$.
   \begin{align*}
      g(p_1 + p_2) &= \begin{pmatrix}
      0 & a_1 + a_2 - 2 \\
      0 & -b_1 -b_2 + c_1 + c_2
   \end{pmatrix} \\
   g(p_1) + g(p_2) &= \begin{pmatrix}
   0 & a_1 - 2 \\
   0 & -b_2 + c_1
\end{pmatrix} + \begin{pmatrix}
   0 & a_2 - 2 \\
   0 & -b_2 + c_2
\end{pmatrix} = \begin{pmatrix}
   0 & a_1 + a_2 - 4 \\
   0 & -b_1 -b_2 + c_1 + c_2
\end{pmatrix} \neq g(p_1 + p_2)
   \end{align*}
   $g$ ist also nicht linear.
\end{itemize}

\subsection{}
Um $Ker(f)$ zu berechnen, muss folgendes Gleichungssystem gelöst werden.
\begin{align*}
   a - c    & = 0 \\
   3b       & = 0 \\
   -2a + 2c & = 0 \\
   \Rightarrow b &= 0 \\
   a &= c
\end{align*}
Also $Ker(f) = \left\{k \cdot \begin{pmatrix} 1 & 0 \\ 0 & 1 \end{pmatrix} \mid k \in \mathbb{R} \right\}$

\subsection{}

Für $\begin{pmatrix} x_1 \\ x_2 \\ x_3 \end{pmatrix}\in \mathbb{R}^3$ gilt 
\begin{align*}
   a - c                                  & = x_1 \\
   3b                                     & = x_2 \\
   -2a + 2c                               & = x_3, \\
   \intertext{was sich umformen lässt zu}
   a                                      & = x_1 + c \\
   3b                                     & = x_2 \\
   -2(x_1 + c) + 2c                       & = x_3 \\
   \Rightarrow -2x_1                      & = x_3
\end{align*}
Es ist daher nicht möglich, $x_1$ und $x_3$ frei zu wählen. Der Vektor
$\begin{pmatrix} 1 & 0 & 1 \end{pmatrix}^T$ wird z.\,B. nicht getroffen. Somit
ist $f$ nicht surjektiv und auch nicht bijektiv.

Der Kern von $f$ ist $\neq \{\mbf{0}\}$. Nach Satz 1.11 ist $f$ also nicht injektiv.

\section{}
\subsection{}

\begin{proof}
   Angenommen, $\{\v{1}, \ldots, \v{n}\}$ sind linear abhängig. Somit existieren
   $\alpha_1, \ldots \alpha_n \in \mathbb{K}$, sodass
   \begin{align*}
      \sum_{i=1}^n \alpha_i \v{i} &= \mbf{0},\\
      \intertext{mit $\exists i : \alpha_i \neq 0$. Anwendung von $f$ ergibt}
      \sum_{i=1}^n \alpha_i f(\v{i}) &= f(\mbf{0}) = \mbf{0}\\
   \end{align*}
   Da nicht $\alpha_i = 0 ~\forall i$, sind $\{f(\v{1}), \ldots, f(\v{n})\}$
   linear abhängig.
\end{proof}

\subsection{}

\begin{proof}
   Angenommen, $\{f(\v{1}), \ldots, f(\v{n})\}$ sind linear abhängig. Somit
   existieren $\beta_1, \ldots, \beta_n \in \mathbb{K}$  mit $\exists i : \beta_i \neq
   0$, sodass
   \begin{align*}
      \beta_1 f(\v{1}) + \ldots + \beta_n f(\v{n}) &= 0 & | -\beta_n f(\v{n})\\
      \beta_1 f(\v{1}) + \ldots + \beta_{n-1} f(\v{n-1}) &= -\beta_n f(\v{n})\\
      f\left(\sum_{i=1}^{n-1}\beta_i \v{i}\right) &=f(-\beta_n \v{n})
   \end{align*}
   Es gilt nun, zwei Fälle zu unterscheinden.
   \begin{itemize}[align=left]
      \item[\bfseries $f$ ist injektiv:] Es gilt wegen der Injektivität
         $\sum_{i=1}^{n-1}\beta_i \v{i} = -\beta_n \v{n}$. Daher
         $\sum_{i=1}^{n}\beta_i \v{i} = \mbf{0}$, wobei nicht
         alle $\beta_i = 0$ sind. Die $\v{i}$ sind also linear abhängig.
      \item[\bfseries $\v{i}$ sind linear unabhängig:] Es gilt wie oben
         \begin{align*}
            \beta_1 f(\v{1}) + \ldots + \beta_n f(\v{n}) &= \mbf{0} \\
            f(\beta_1 \v{1} + \ldots + \beta_n \v{n}) &= \mbf{0} \\
            \intertext{mit $\beta_1 \v{1} + \ldots + \beta_n \v{n} \neq \mbf{0}$, da
            nicht alle $\beta_i = 0$ sind. Allerdings ist auch}
            f(\mbf{0}) &= \mbf{0}, \\
            \intertext{denn $f$ ist linear. $f$ kann also nicht injektiv sein,
            da zwei ungleiche $\v{i} \in V$ auf das selbe Element in $W$
         abgebildet werden.}
         \end{align*}
   \end{itemize}
   Falls also $f(\v{1}), \ldots, f(\v{n})$ linear abhängig sind, sind entweder
   $\v{1}, \ldots, \v{n}$ linear abhängig, oder $f$ ist nicht injektiv, ein
   Widerspruch.
\end{proof}

\subsection{}

Für Aussagen $A, B, C$ gilt 
\begin{align*}
   & A \wedge B  \rightarrow C \\
   \Leftrightarrow & \neg (A \wedge B)  \rightarrow C \\
   \Leftrightarrow & \neg A \vee \neg B  \vee C \\
   \Leftrightarrow & \neg B  \vee \neg (A \wedge \neg C) \\
   \Leftrightarrow & A \wedge \neg C  \rightarrow \neg B
\end{align*}

Die Aussage ist also logisch äquivalent zu folgender:
\begin{equation*}
   span(\v{1}, \ldots, \v{n}) = V \wedge \neg f~\text{injektiv} \Rightarrow \neg
   f(\v{1}, \ldots, \v{n}) ~\text{linear unabhängig.}
\end{equation*}

\begin{proof}
Seien $\mbf{u}_1, \mbf{u}_2 \in V$ gegeben mit $\mbf{u}_1 \neq \mbf{u}_2$ und
$f(\mbf{u}_1) =
f(\mbf{u}_2)$. Nach Vorraussetzung existieren Koeffizienten $\alpha_i, \beta_i \in
\mathbb{K}$ mit 
\begin{align*}
   \alpha_1 \v{1} + \ldots + \alpha_n \v{n} &= \mbf{u}_1 \\
   \beta_1 \v{1} + \ldots + \beta_n \v{n} &= \mbf{u}_2 \\
   \intertext{Also:}
   f(\alpha_1 \v{1} + \ldots + \alpha_n \v{n}) &= f(\mbf{u}_1) = f(\mbf{u}_2) =
   f(\beta_1 \v{1} + \ldots + \beta_n \v{n}) \\
   f\left(\sum \alpha_i \v{i}\right) - f\left(\sum \beta_i \v{i}\right) &= \mbf{0} \\
   \sum (\alpha_i - \beta_i)f(\v{i}) &= \mbf{0}
\end{align*}
Also sind die $f(\v{i})$ linear abhängig.
\end{proof}

\section{}
\subsection{}
$Ker(\varphi)$ ist der Lösungsraum des Gleichungssystems
\begin{align}
   2x_1     +  x_3  &=  0 \\
   2x_1  +  x_2    &=  0 \\
   4x_1  +  x_2  +  x_3  &=  0 \\
   \intertext{Subtraktion von (3) - (1) - (2) ergibt}
   \nonumber 2x_1     +  x_3  &=  0 \\
   \nonumber 2x_1  +  x_2    &=  0 \\
   \nonumber 0      &=  0 \\
   \intertext{und daher}
   \nonumber x_1 &= -\frac{1}{2}x_3 \\
   \nonumber x_1 &= -\frac{1}{2}x_2 \\
   \nonumber x_2 &= x_3
\end{align}
Also ist $Ker(\varphi) = \left\{\begin{pmatrix} -0.5k & k & k \end{pmatrix}^T \mid k \in
\mathbb{R}\right\}$. Offensichtlich ist $dim(Ker(\varphi)) = 1$.

\subsection{}

$dim V = dim_\mathbb{R}(Bild(\varphi)) + dim(Ker(\varphi)) \Rightarrow dim(Bild(\varphi)) = 2$

\subsection{}

Eine Basis von $Bild(\varphi)$ ist die maximale Zahl linear unabhängiger Spalten
von $A$ (siehe Definition 1.9), also beispielsweise 
\begin{equation*}
   \left\{
      \begin{pmatrix}
         0 \\ 1 \\ 1
      \end{pmatrix},
      \begin{pmatrix}
         1 \\ 0 \\ 1
      \end{pmatrix}
   \right\}
\end{equation*}

\section{}

\subsection{$f$ ist linear \& Bestimmung der Matrix}

Da Addition, Subtraktion und Multiplikation in $\mathbb{C}$ kommutativ und
distributiv sind, ist die Lienarität eigentlich trivial. Wir rechnen nach:

Seien $(z_1, z_2, z_3)$ und $(y_1, y_2, y_3) \in \mathbb{C}^3$.
\begin{align*}
   f(z_1 + y_1, z_2 + y_2, z_3 + y_3) & = (z_1 + y_1 +i(z_2 + y_2) - (z_3 + y_3), i(z_1 + y_1) - (z_2 + y_2) + (1+i)(z_3 + y_3)) \\
                                      & = (z_1 + y_1 + iz_2 + iy_2 - z_3 - y_3, iz_1 + iy_1 - z_2 - y_2 + (1+i)z_3 + (1+i)y_3)   \\
                                      & = (y_1 + iy_2 - y_3, iy_1 - y_2 + (1+i)y_3) + (y_1 + iy_2 - y_3, iy_1 - y_2 + (1+i)y_3)  \\
                                      & = f(y_1, y_2, y_3) + f(y_1, y_2, y_3)
\end{align*}

Sei zudem $\alpha \in \mathbb{C}$.
\begin{align*}
   f(\alpha z_1, \alpha z_2, \alpha z_3) & = (\alpha z_1, \alpha i z_2 - \alpha z_3, \alpha i z_1 - \alpha z_2 + \alpha (1+i)z_3) \\
                                         & = \alpha f(z_1, z_2, z_3)
\end{align*}

Die Matrix $A_f$ muss offensichtlich $\in \mathbb{C}^{2 \times 3}$ sein.
Aus der Abbildungsvorschrift lässt sich sofort ablesen.
\begin{align*}
A_f &= \begin{pmatrix} 1 & i & -1 \\ i & -1 & (1+i) \end{pmatrix}
\xrightarrow{II - i\cdot I} \begin{pmatrix} 1 & i & -1 \\ 0 & 0 & 1 +2i\end{pmatrix}
\end{align*}

\subsection{Kern, Bild \& Dimensionen}

Der Kern von $f$ ist der Lösungsraum von $A_f\mbf{x} = 0$:
\begin{align*}
   x_1 + ix_2 -x_3 & = 0     \\
   (1+2i)x_3       & = 0     \\
   \Rightarrow x_3 & = 0     \\
   \Rightarrow x_1 & = -ix_2
\end{align*}

\newcommand{\inlinepmatrix}[1]{\begin{pmatrix}#1\end{pmatrix}}

\begin{equation*}
   Ker(f) = \left\{k \begin{pmatrix}-i & 1 & 0\end{pmatrix}^T \mid k \in
\mathbb{C} \right\} \text{ und } dim_\mathbb{C}(Ker(f)) = 1.
\end{equation*}

\begin{equation*}
   Bild(f) = span\left\{\inlinepmatrix{i \\ -1}, \inlinepmatrix{-1 \\
   1+i}\right\},
\end{equation*}
denn die ersten zwei Spalten von $A_f$ sind linear abhängig.
Also $dim_\mathbb{C}(Bild(f)) = 2$.
\end{document}
