\documentclass{../mfa}
\sheet{1}
\begin{document}
\maketitle

\section{}
\subsection{}
Wir nehmen an, $\mbf{v}_1$ und $\mbf{v}_2$ seien linear abhängig. Folglich existieren
$\alpha_1,~\alpha_2 \in \mathbb{K}$, sodass $\alpha_1 \mbf{v}_1 + \alpha_2 \mbf{v}_2 = \mbf{0}$
\begin{align}
   \alpha_1 \mbf{v}_1 + \alpha_2 \mbf{v}_2 &= \mbf{0} & \hspace{1cm} & | \cdot A
   \\
   \alpha_1 A \mbf{v}_1 + \alpha_2 A \mbf{v}_2 &= \mbf{0} \\
   \alpha_1 \lambda_1 \mbf{v}_1 + \alpha_2 \lambda_2 \mbf{v}_2 &= \mbf{0} & & | \mbf{v}_i~\text{Eigenvektoren}
\end{align}

Multipliziert man (1) mit $\lambda_1$, so erhält man \setcounter{equation}{5}
\begin{align}
   \alpha_1 \lambda_1 \mbf{v}_1 + \alpha_2 \lambda_1 \mbf{v}_2 &= \mbf{0} \\
   \intertext{Subtraktion (1) $-$ (6) ergibt}
   \mbf{0} + (\lambda_2 - \lambda_1)\alpha_2 \mbf{v}_2 &= \mbf{0}
\end{align}
Es muss also entweder (a) $\lambda_1 = \lambda_2$ oder (b) $\alpha_2 = 0$ oder
(c) $\mbf{v}_2 = \mbf{0}$ gelten. (a) gilt nicht nach Aufgabenstellung. (b) gilt nicht
aufgrund der Definition der linearen Unabhängigkeit. (c) gilt nicht aufgrund der
Definition eines Eigenvektors. Aus dem Widerspuch folgt, dass $\mbf{v}_1$ und
$\mbf{v}_2$ linear unabhängig sein müssen.
   
\end{document}
